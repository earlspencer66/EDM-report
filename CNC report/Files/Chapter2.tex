\chapter{Literature Review}
\lhead{\leftmark}
\label{sec:review}
\section{Numerical Control in Machining}
Numerical Control (NC) is a non-conventional machine control method. A computer program is used to control the machine tool rather than a human operator manually controlling the machining parameters such as the speeds, feeds and depth of cut, as well as the part dimensions. Repeatability and quality are greatly improved over conventional machines. The use of NC machines also reduces non-machining times, such as setup times, tool change times and change of cutting speeds and feeds. It also relieves the operator of tasks such as changing machining parameters(cuttingg speeds and feeds), and locating the tool relative to the work. Even the most simple NC forms, and digital readout equipment, provide much greater accuracy and productivity.
\section{CNC Machining}
Computer Numerical Control is an NC method where an onboard microprocessor is directly programmed to control the machine tool. Early NC machines used punched cards for control, while CNC machines use software programming to achieve the machine control.
\subsection{The CNC Machining System}
The main components of the CNC system are shown in Fig. \ref{fig:cnc}. These components include:
\begin{enumerate}
	\item The Machine Tool Unit - This is responsible for the extrication work. It consists of the tools to be used, the attachment mechanism, the spindle and all moving parts involved in machining.
	\item The Machine Tool Drives - Classified as either spindle or feed drives. \begin{enumerate}
		\item Spindle drives rotate over a wide range of velocities, measured in rpm.
		\item Feed drives convert angular motion of the motors to linear transverse speeds, measured in mm/min.
	\end{enumerate}
	\item The CNC unit - Runs under a program known as the CNC executive which translates programs written in internationally recognized standard language.
\end{enumerate}
\begin{figure}[h!]
	\centering
	\includegraphics[width=0.8\linewidth]{Figures/cncschematic}
	\caption[CNC Machining System Schematic]{CNC Machine Table Location Schematic}
	\label{fig:cnc}
\end{figure}
\subsection{CNC Machining Process Parameters}
The CNC Machining Process can be used to machine
The performace of a CNC Machining operation is determined by a number of properties:
\begin{enumerate}
	\item Material Removal Rate (MRR)
	\item Surface Quality
	\item Accuracy
\end{enumerate}
These are determined by the process parameters such as:
\begin{itemize}
	\item Pulse characteristics
	\item Workpiece properties - ductility, hardness, grain formation
	\item Use of cutting fluids such as 
	\item Dielectric properties
	\item Tool electrode - material, movement, wear
\end{itemize}

\subsection{Applications}
CNC Machining can be employed in:
\begin{enumerate}
	\item Micro-EDM: Micromachining of holes, slots and dies.
	\item EDM drilling - creation of cooling channels in turbines made of hard alloys.
	\item Electrodischarge sawing where billets and bars are created.
	\item Machining spheres, dies and moulds.
	\item EDM of ceramics used in insulation.
	\item Texturing - texturing is applied to the steel sheets during the final stages of cold rolling.
\end{enumerate}
\section{Other NC Machining Processes}
Numerical control has been applied to a wide variety of other production processes\cite{Black2011}, some of which are listed below
\begin{enumerate}
	\item NC Punches - Numerical control is used for \textit{X-Y} control on the table.
	\item CNC Wire EDM Machines
	\item Laser and water-jet abrasive machining
	\item Flame cutters
\end{enumerate}
\subsection{Advantages and Disadvantages of Numerical Control}
Numerica Control has several advantages as compared to other conventional methods of control\cite{Kalpakjian2010}, some of which are:
\begin{enumerate}
	\item Higher production rates, productivity, and product quality; greater operational flexibility; the capacity to make complicated forms with good dimensional precision and repeatability; and reduced scrap loss.
	\item Making machine adjustments is simple.
	\item With each setup, more operations can be completed, and the setup and machining lead times are less than with traditional methods.
	\item Programs can be quickly created and retrieved at any moment.
	\item The level of operator skill needed is lower than that of a skilled machinist, giving the operator more time to focus on other activities around the workspace.
\end{enumerate}
Some of the major limitation of NC machining are:
\begin{enumerate}
	\item Initially expensive cost of the equipment.
	\item The cost of programming and the price of computer time.
	\item The unique maintenance needed.
	\item Preventive maintenance is essential since these equipment are complicated systems and breakdowns can be expensive.
\end{enumerate}