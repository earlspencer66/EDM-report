\section{Introduction}
\lhead{\leftmark}
\label{sec:introduction}
\subsection{Electrodischarge Machining}
The principle of electrical-discharge machining (EDM) (also called electrodischarge or spark-erosion machining) is based on the erosion of metals by spark discharges \cite{classnotes}. When two current-conducting wires are allowed to touch each other, an arc is produced. A closer look at where the two wires meet, we can see that a small amount of the metal has been worn away, forming a tiny crater.\\
Despite the fact that this phenomenon has been well known since the invention of electricity, a machining method based on that idea wasn't established until the 1940s. The EDM process has become one of the most important and widely used production technologies in manufacturing\cite{youssef2020non}. EDM is used in applications that require a high degree of accuracy\cite{youssef2020non}.
\subsection{Objectives}
\begin{enumerate}
\item To create a program to manufacture a part using Electro-discharge Machining.
\item To create and examine a part made from the Electro-discharge Machining process.
\end{enumerate}
